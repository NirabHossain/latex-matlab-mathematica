\documentclass[11pt]{article}
\usepackage{amsmath}
\usepackage{amsfonts}
\usepackage{amssymb}
\usepackage{yfonts}
\usepackage[utf8]{inputenc}
\usepackage[english]{babel}

\newtheorem{theorem}{Theorem}

\begin{document}
	\title{Assignment 01}
	\author{Md. Abul Kashem\\
		 SM-020-053}
	
	\date{\today}
	\maketitle
	
	\flushleft \huge \textbf{\LaTeXe Exercises}\\
	\Large \textbf{Text Exercises}\\
	\large \textbf{Easy}
	\begin{enumerate}
		\item A simple test sentence: the quick brown fox jumps over the lazy dog.
		
		\item \textbf{Bold} and \textit{italic} fonts may be used to emphasis to the text. It is also possible to use \textsf{sans-serif} and \texttt{typewriter-style} fonts.
		
		\item The \LaTeX language uses some special characters that must be preceded by a \textbackslash or they will not be printed. These include \$ \& \% \# \{ \} \textasciitilde \textasciicircum 
		
		\item Leaving a blank line between sentences marks a break between paragraphs. \\
		A new paragraph should contain a new idea, of course. 
		
		\item It's possible.\\
		to break the lines\\
		wherever you like. You can move the text\\
		\hspace*{3 cm}horizontally using the \textbackslash hsapce* command. (The gap is 3 cm in this case).\vspace*{1.5 cm} \\
		You can also move the text vertically using the \textbackslash vspace* command. (Now the gap is 1.5 cm). This only works between paragraphs.
		
		\item Font size can be varied from \tiny tiny \scriptsize up \footnotesize to \small the \normalsize normalsize \large and \Large then \LARGE up \huge to \Huge Huge.\normalsize \textit{This is an example in which font size matters.} 
	\end{enumerate}
	
	\large \textbf{Medium}
	
	\begin{enumerate}
		
		\item \LaTeXe  uses environments to perform useful functions; for example,\\
		\begin{center}
			center (note US spelling) environment,
		\end{center}
		flushleft environment,\\
		
		\item Environments can also be used to make lists:
		
		\begin{itemize}
			\item itemize does not number list entries
			\item bullet points are used
		\end{itemize}
		
		\begin{enumerate}	
			\item[1.] enumerate does number the entries
			\item[2.] in fact, enumerate was used to generate the example numbers on this sheet.
		\end{enumerate}
		
%		\item \texttt{In the verbatim environment, text will be printed directly \textbackslash emph\{latex commands will not be executed\} \hspace*{3cm} and spaces are important.}
		
		\item Tables can also be generated easily using environments:\\
		\centering
		\begin{tabular}{ll}
			1.0 &One\\
			2.0 &Two\\
			3.0 &Three\\
		\end{tabular}
	\end{enumerate}
	
	\large \textbf{Tricky Stuff}
	
	\begin{enumerate}
		\item Quick tricky tables can be constructed\\
		\begin{tabular}{|r|c|l|}
			\hline
			\multicolumn{3}{|c|}{\textbf{Famous Dead Mathematicians}}\\
			\hline
			\textit{Name} &\textsf{Fields of study}&\texttt{Survives}\\
			\hline
			Archimedes&Geometry, Bath water & A Principle, An axiom\\
			&Ways of killing Romans &\\
			\hline
			Euler&You name it,&An equation,A constant,\\
			&he studied it&A formula, A method\\
			\hline
			\huge Gauss&\normalsize integrations, integers& A distribution, A theorem\\
			\hline
		\end{tabular}
		
		\item You\vspace*{-0.2 cm}\\
		\hspace*{1 cm}can also\vspace*{-0.2 cm}\\
		\hspace*{2.5 cm}make beautiful patterns\vspace*{-0.2 cm}\\
		\hspace*{3.5 cm}with text\vspace*{-0.2 cm}\\
		\hspace*{2.5 cm}but then again\\
		\hspace*{1 cm} \huge Why?\\
	\end{enumerate}
	
	
	
	\section*{Mathematical Exercise}
	\large \textbf{Easy}
	
	\begin{enumerate}
		\item Any equation can be directly inserted into text, $x^2+1=0$
		\item Longer (or taller) equations can be best inserted using the equation enviornment.
		\begin{equation}
		\int \frac{x^2+3x+1}{2x+7}\hspace{1 pt}dx
		\end{equation}
		An advantage that your equations will be automatically numbered.
		
		\item Traditional mathematics typesetting demands that variables are italicised and this is the default in math-mode. The \textbackslash \texttt{mbox} or \textbackslash \texttt{text} (part of the amsmath package) commands must be used to generate normal text. Compare\\ 
		\begin{equation}
		a=b+c \hspace{0.5 cm}if  b>c,
		\end{equation}
		to \\
		\begin{equation}
		a=b+c \hspace{0.5 cm} \mbox{if } b>c
		\end{equation}
		
		\item There are some special commands function names
		$$\sin^2x+\cos^2x=1, \hspace*{0.3 cm} f''(x)=\ln x $$
		
		\item Lots of mathematical symbols are easily accessible
		$$\Upsilon \notin [1,\infty), \hspace*{0.2 cm} R\propto C^{\frac{1}{2}},\hspace*{0.2 cm} \mbox{as}\hspace*{0.1 cm} C\to \infty,\hspace*{0.2 cm} \sum_{k=1}^{\infty} \frac{1}{k^2}=\frac{\pi^2}{12}.  $$
		
		\item Vectors may be denoted using the \textbackslash \texttt{boldmath} command; i.e. the vector,\boldmath $x$ \unboldmath. Boldmath remains on until turned off with the \textbackslash \texttt{unboldmath} command. Check this now $a^2+b^k=c^k$
		
		\item Brackets change size automatically
		$$\left(A+B\right), \mbox{ is smaller than} \left[\frac{A+B}{C+D}\right]